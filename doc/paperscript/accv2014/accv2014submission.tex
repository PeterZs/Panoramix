% Updated in September 2012 by In Kyu Park
% Updated in April 2002 by Antje Endemann, ...., and in September 2010 by Reinhard Klette
% Based on CVPR 07 and LNCS style, with modifications by DAF, AZ and elle 2008, AA 2010, ACCV 2010, ACCV 2012

\documentclass[runningheads]{llncs}
\usepackage{graphicx}
\usepackage{amsmath,amssymb} % define this before the line numbering.
\usepackage{lineno}
\usepackage{color}

%===========================================================
\begin{document}

%macro for raising the point in decimal numbers; see example in the abstract
\newcommand{\point}{
    \raise0.7ex\hbox{.}
    }

%Do   -- NOT --    use any additional macros

\pagestyle{headings}

\mainmatter

%===========================================================
\title{3D Indoor Reconstruction from Panoramic Image} % Replace with your title

\titlerunning{3D Indoor Reconstruction from Panoramic Image} % Replace with your title

\authorrunning{Hao Yang, Hui Zhang} % Replace with your names

\author{Hao Yang, Hui Zhang} % Replace with your names
\institute{Institude of CAD and Computer Graphics, Tsinghua University} % Replace with your institute's address

\maketitle

%===========================================================
\begin{abstract}
% The abstract should summarize the contents of the paper and should
% contain at least 70 and at most 300 words. It should be set in 9-point
% font size and should be inset $1\point0$~cm from the right and left margins.

% Please follow the instructions as outlined below. This will save time for all involved.
% We aim at a uniform appearance of the proceedings.
\end{abstract}

%===========================================================
\section{Introduction}

// Current approaches for recovering 3D structures from 2D image(s):
// 1. Structure from Motion (SfM) or Stereo Vision
//    requires: video/multiple images with varied eye (projection center) positions
//    time consuming/not applicable in large scale scenes since the user must MOVE
// 2. Reconstruction from a single image
//    restrictions to scenes, time consuming, relatively poor performance
// 

\section{Related Work}

// Structure from Motion (SfM) or Stereo Vision
// ...
// Reconstruction from a single image
// ...
// Indoor scene understanding from a single image
// ...
// ... from multiple panoramas (CVPR2014)

\section{3D Indoor Model Reconstruction from one Panoramic Image}

%-------------------------------------------------------------------------
\subsection{Overview of the Proposed Method}



%-------------------------------------------------------------------------
\subsection{3D Line Segments Recovery}

\subsubsection {Geometric Constraints}

// Locate 3 orthogonal vanishing points;
// Classify lines
// Locate other horizontal vanishing points
// -> Detect horizontal (but not classified) lines
// -> Locate perpendicularities among horizontal lines

// Detect symmetric line pairs on vertical planes
// HOW?

\subsubsection {Reconstruct Line Segment Parts}

// Constraints on single line:
//   class of line

// Detect line junctions
//   I, X junction (incidence)
//   L, X, Y, W junction (intersection)
//   T junction (occludeness)

// Constraints between adjoint lines
//   incidence, intersection
//   geometric constraint between two lines (horizontal orthogonality, symmetry)

// Reconstruct line segment parts
// result: locally constrained lines, several connected components, no face construction


%-------------------------------------------------------------------------
\subsection{Region Reasoning for Face Construction}

// use regions to stitch line segment parts and construct faces

\subsubsection {Boundary Regions and Region-Graph} % regions connecting isolated(without constraints) lines

// Assumption:
// 1. Pixels in each region satisfy piecewise planarity (one region one plane)
// 2. Occluding edges are visualized as T junctions

// Labeling regions: horizontal? vertical? neither horizontal nor vertical?


// Constraints on single region:
//  regions containing (close to) vertical lines (if not occluding lines) should be vertical 
//  regions containing (close to) horizontal lines should be ...
//	other region features ...

// Constraints between adjacent regions:
//   three kinds of Relation between adjacent regions:
//   1. Continuity, same orientation label, and connected
//   2. Connectivity, different orientation label but connected
//   3. Disconnectivity, not connected



\subsubsection {Graph Optimization}

// Design energy function
//  0. imagine this:
//       treat each region as a patch/ a piece of cloth with some elasticity, 
//       if l1, l2, ... l_{n_r} are lines contained in region r,
//       then pin r with those corresponding reconstructed lines in 3D space
//       well 
//  1. punish disconnectivities between adjacent regions
//  2. punish (slightly) discontinuities between adjacent regions
//  3. punish spatial inconsistencies between reconstructed region faces and reconstructed line segments

%-------------------------------------------------------------------------
\section{Experimental Results}

\subsection {}



%===========================================================
\bibliographystyle{splncs}

\begin{thebibliography}{1}

\bibitem{Alpher02}
Alpher, A.:
Advances in Frobnication.
J. of Foo
\textbf{12} (2002)  234--778

\bibitem{Alpher03}
Alpher, A., Fotheringham-Smythe, J.P.N.:
Frobnication revisited.
J. of Foo
\textbf{13} (2003)  234--778

\bibitem{Herman04}
Herman, S., Fotheringham-Smythe, J.P.N., Gamow, G.:
Can a machine frobnicate?
J. of Foo
\textbf{14} (2004)  234--778

\bibitem{Smith09}
Smith, F.:
{\it The Frobnicatable Foo Filter}.
GreatBooks, Atown (2009)

\bibitem{Wills99}
Wills, H.:
Frobnication tutorial.
Technical report CS-1204, XYZ University, Btown (1999)

\end{thebibliography}

%this would normally be the end of your paper, but you may also have an appendix
%within the given limit of number of pages
%\end{document}

%===========================================================
\clearpage\mbox{}Page \thepage\ of the manuscript.
\clearpage\mbox{}Page \thepage\ of the manuscript.
\clearpage\mbox{}Page \thepage\ of the manuscript.
\clearpage\mbox{}Page \thepage\ of the manuscript.
12 pages are without extra charge for additional pages.
\par\vfill\par
This is the standard number of pages required for ACCV 2012.

\clearpage\mbox{}Page \thepage\ of the manuscript.
\clearpage\mbox{}Page \thepage\ of the manuscript.
Pages 13 and 14 are ``additional pages''; they will need extra payment.
\par\vfill\par
Now we have reached the maximum size of the ACCV 2012 final paper.

\end{document}
